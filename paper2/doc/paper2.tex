%
% FH Technikum Wien
% !TEX encoding = UTF-8 Unicode
%
% Erstellung von Master- und Bachelorarbeiten an der FH Technikum Wien mit Hilfe von LaTeX und der Klasse TWBOOK
%
% Um ein eigenes Dokument zu erstellen, müssen Sie folgendes ergänzen:
% 1) Mit \documentclass[..] einstellen: Master- oder Bachelorarbeit, Studiengang und Sprache
% 2) Mit \newcommand{\FHTWCitationType}.. Zitierstandard festlegen (wird in der Regel vom Studiengang vorgegeben - bitte erfragen)
% 3) Deckblatt, Kurzfassung, etc. ausfüllen
% 4) und die Arbeit schreiben (die verwendeten Literaturquellen in Literatur.bib eintragen)
%
% Getestet mit TeXstudio mit Zeichenkodierung ISO-8859-1 (=ansinew/latin1) und MikTex unter Windows
% Zu beachten ist, dass die Kodierung der Datei mit der Kodierung des paketes inputenc zusammen passt!
% Die Kodierung der Datei twbook.cls MUSS ANSI betragen!
% Bei der Verwendung von UTF8 muss dnicht nur die Kodierung des Dokuments auf UTF8 gestellt sein, sondern auch die des BibTex-Files!
%
% Bugreports und Feedback bitte per E-Mail an latex@technikum-wien.at
%
% Versionen
% *) V0.7: 9.1.2015, RO: Modeline angepasst und verschoben
% *) V0.6: 10.10.2014, RO: Weitere Anpassung an die UK
% *) V0.5: 8.8.2014, WK: Literaturquellen überarbeitet und angepasst
% *) V0.4: 4.8.2014, WK: Initalversion in SVN eingespielt
%
\documentclass[MIS,Projekt,english]{twbook}%\documentclass[Bachelor,BMR,ngerman]{twbook}
\usepackage[utf8]{inputenc}
\usepackage[T1]{fontenc}

%
% Bitte in der folgenden Zeile den Zitierstandard festlegen
\newcommand{\FHTWCitationType}{IEEE} % IEEE oder HARVARD möglich - wenn Sie zwischen IEEE und HARVARD wechseln, bitte die temorären Dateien (aux, bbl, ...) löschen
%
\ifthenelse{\equal{\FHTWCitationType}{HARVARD}}{\usepackage{harvard}}{\usepackage{bibgerm}}

% Definition Code-Listings Formatierung:
\usepackage[final]{listings}
\lstset{captionpos=b, numberbychapter=false,caption=\lstname,frame=single, numbers=left, stepnumber=1, numbersep=2pt, xleftmargin=15pt, framexleftmargin=15pt, numberstyle=\tiny, tabsize=3, columns=fixed, basicstyle={\fontfamily{pcr}\selectfont\footnotesize}, keywordstyle=\bfseries, commentstyle={\color[gray]{0.33}\itshape}, stringstyle=\color[gray]{0.25}, breaklines, breakatwhitespace, breakautoindent}
\lstloadlanguages{[ANSI]C, C++, [gnu]make, gnuplot, Matlab}

%Formatieren des Quellcodeverzeichnisses
\makeatletter
% Setzen der Bezeichnungen für das Quellcodeverzeichnis/Abkürzungsverzeichnis in Abhängigkeit von der eingestellten Sprache
\providecommand\listacroname{}
\@ifclasswith{twbook}{english}
{%
    \renewcommand\lstlistingname{Code}
    \renewcommand\lstlistlistingname{List of Code}
    \renewcommand\listacroname{List of Abbreviations}
}{%
    \renewcommand\lstlistingname{Quellcode}
    \renewcommand\lstlistlistingname{Quellcodeverzeichnis}
    \renewcommand\listacroname{Abkürzungsverzeichnis}
}
% Wenn die Option listof=entryprefix gewählt wurde, Definition des Entyprefixes für das Quellcodeverzeichnis. Definition des Macros listoflolentryname analog zu listoflofentryname und listoflotentryname der KOMA-Klasse
\@ifclasswith{scrbook}{listof=entryprefix}
{%
    \newcommand\listoflolentryname\lstlistingname
}{%
}
\makeatother
\newcommand{\listofcode}{\phantomsection\lstlistoflistings}

% Die nachfolgenden Pakete stellen sonst nicht benötigte Features zur Verfügung
\usepackage{blindtext}

%
% Einträge für Deckblatt, Kurzfassung, etc.
%
\title{Projekt 1\\Paper 2}
\author{}
%\studentnumber{XXXXXXXXXXXXXXX}
%\author{Titel Vorname Name, Titel\and{}Titel Vorname Name, Titel}
%\studentnumber{XXXXXXXXXXXXXXX\and{}XXXXXXXXXXXXXXX}
\supervisor{Titel Vorname Name, Titel}
%\supervisor[Begutachter]{Titel Vorname Name, Titel}
%\supervisor[Begutachterin]{Titel Vorname Name, Titel}
%\secondsupervisor{Titel Vorname Name, Titel}
%\secondsupervisor[Begutachter]{Titel Vorname Name, Titel}
%\secondsupervisor[Begutachterinnen]{Titel Vorname Name, Titel}
\place{Wien}
\kurzfassung{\blindtext}
\schlagworte{Schlagwort1, Schlagwort2, Schlagwort3, Schlagwort4}
\outline{\blindtext}
\keywords{Keyword1, Keyword2, Keyword3, Keyword4}
%\acknowledgements{\blindtext}

\begin{document}

%Festlegungen für den HARVARD-Zitierstandard
\ifthenelse{\equal{\FHTWCitationType}{HARVARD}}{
\bibliographystyle{Harvard_FHTW_MR}%Zitierstandard FH Technikum Wien, Studiengang Mechatronik/Robotik, Version 1.2e
\citationstyle{dcu}%Correct citation-style (Harvardand, ";" between citations, "," between author and year)
\citationmode{abbr}%use "et al." with first citation
\iflanguage{ngerman}{
    %Deutsch Neue Rechtschreibung
    \newcommand{\citepic}[1]{(Quelle: \protect\cite{#1})}%Zitat: Bild
    \newcommand{\citefig}[2]{(Quelle: \protect\cite{#1}, S. #2)}%Zitat: Bild aus Dokument
    \newcommand{\citefigm}[2]{(Quelle: modifiziert "ubernommen aus \protect\cite{#1}, S. #2)}%Zitat: modifiziertes Bild aus Dokument
    \newcommand{\citep}{\citeasnoun}%In-Line Zitiat entweder mit \citep{} oder \citeasnoun{}
    \newcommand{\acessedthrough}{Verf{\"u}gbar unter:}%Für URL-Angabe
    \newcommand{\acessedthroughp}{Verf{\"u}gbar bei:}%Für URL-Angabe (Geschützte Datenbank, Zugriff durch FH)
    \newcommand{\acessedat}{Zugang am}%Für URL-Datum-Angabe
    \newcommand{\singlepage}{S.}%Für Seitenangabe (einzelne Seite)
    \newcommand{\multiplepages}{S.}%Für Seitenangabe (mehrere Seiten)
    \newcommand{\chapternr}{K.}%Für Kapitelangabe
    \renewcommand{\harvardand}{\&}%Harvardand in Zitaten
    \newcommand{\abstractonly}{ausschließlich Abstract}
    \newcommand{\edition}{. Auflage}%Angabe der Auflage
}{
\iflanguage{german}{
    %Deutsch
    \newcommand{\citepic}[1]{(Quelle: \protect\cite{#1})}%Zitat: Bild
    \newcommand{\citefig}[2]{(Quelle: \protect\cite{#1}, S. #2)}%Zitat: Bild aus Dokument
    \newcommand{\citefigm}[2]{(Quelle: modifiziert "ubernommen aus \protect\cite{#1}, S. #2)}%Zitat: modifiziertes Bild aus Dokument
    \newcommand{\citep}{\citeasnoun}%In-Line Zitiat entweder mit \citep{} oder \citeasnoun{}
    \newcommand{\acessedthrough}{Verf{\"u}gbar unter:}%Für URL-Angabe
    \newcommand{\acessedthroughp}{Verf{\"u}gbar bei:}%Für URL-Angabe (Geschützte Datenbank, Zugriff durch FH)
    \newcommand{\acessedat}{Zugang am}%Für URL-Datum-Angabe
    \newcommand{\singlepage}{S.}%Für Seitenangabe (einzelne Seite)
    \newcommand{\multiplepages}{S.}%Für Seitenangabe (mehrere Seiten)
    \newcommand{\chapternr}{K.}%Für Kapitelangabe
    \renewcommand{\harvardand}{\&}%Harvardand in Zitaten
    \newcommand{\abstractonly}{ausschließlich Abstract}
    \newcommand{\edition}{. Auflage}%Angabe der Auflage
}{
    %Englisch
    \newcommand{\citepic}[1]{(Source: \protect\cite{#1})}%Zitat: Bild
    \newcommand{\citefig}[2]{(Source: \protect\cite{#1}, p. #2)}%Zitat: Bild aus Dokument
    \newcommand{\citefigm}[2]{(Source: taken with modification from \protect\cite{#1}, p. #2)}%Zitat: modifiziertes Bild aus Dokument
    \newcommand{\citep}{\citeasnoun}%In-Line Zitiat entweder mit \citep{} oder \citeasnoun{}
    \newcommand{\acessedthrough}{Available at:}%Für URL-Angabe
    \newcommand{\acessedthroughp}{Available through:}%Für URL-Angabe (Geschützte Datenbank, Zugriff durch FH)
    \newcommand{\acessedat}{Accessed}%Für URL-Datum-Angabe
    \newcommand{\singlepage}{p.}%Für Seitenangabe (einzelne Seite)
    \newcommand{\multiplepages}{pp.}%Für Seitenangabe (mehrere Seiten)
    \newcommand{\chapternr}{Ch.}%Für Kapitelangabe
    \renewcommand{\harvardand}{\&}%Harvardand in Zitaten
    \newcommand{\abstractonly}{Abstract only}
    \newcommand{\edition}{~edition}%Edition -> note, that you have to write "edition = {2nd},"!
}}}

\maketitle

%
% .. und hier beginnt die eigentliche Arbeit. Viel Erfolg beim Verfassen!
%
\chapter{Erste Überschrift der Ebene 1 (chapter)}
\blinddocument

\blindmathpaper

\section{Erste Überschrift Tiefe 2 (section)}
\blindtext

\subsection{Erste Überschrift Tiefe 3 (subsection)}
\blindtext

\subsubsection{Erste Überschrift Tiefe 4 (subsubsection)}
\blindtext

\chapter{Zweite Überschrift der Tiefe 1 (chapter)}
\blindtext

\section{Zweite Überschrift Tiefe 2 (section)}
\blindtext

\section{Zweite Überschrift Tiefe 2 (section)}
\blindtext

\subsection{Zweite Überschrift Tiefe 3 (subsection)}
\blindtext

\subsection{Dritte Überschrift Tiefe 3 (subsection)}
\blindtext

\subsubsection{Zweite Überschrift Tiefe 4 (subsubsection)}
\blindtext

\noindent Querverweise werden in \LaTeX{} automatisch erzeugt und verwaltet, damit sie leicht aktualisiert werden können. Hier wird zum Beispiel auf Abbildung \ref{Abb1} verwiesen.

\begin{figure}[!htbp]
\centering
\includegraphics[width=0.5\linewidth]{PICs/buchruecken}
\caption{Beispiel für die Beschriftung eines Buchrückens.}\label{Abb1}
\end{figure}
\begin{figure}[!htbp]
\centering
\includegraphics[width=0.5\linewidth]{PICs/buchruecken}
\caption{2. Beispiel für die Beschriftung eines Buchrückens.}\label{Abb2}
\end{figure}

Und hier ist ein Verweis auf Tabelle \ref{tab1}. Das gezeigte Tabellenformat ist nur ein Beispiel. Tabellen können individuell gestaltet werden.

\begin{table}[!htbp]
\centering
\caption{Semesterplan der Lehrveranstaltung \glqq Angewandte Mathematik\grqq.}\label{tab1}
\begin{tabular}{| p{0.3\linewidth} | p{0.3\linewidth} | p{0.3\linewidth} |}\hline
Datum & Thema & Raum\\\hline
20.08.2008 & Graphentheorie & HS 3.13\\
01.10.2008 & Biomathematik & HS 1.05\\\hline
\end{tabular}
\end{table}
\begin{table}[!htbp]
\centering
\caption{2. Semesterplan der Lehrveranstaltung \glqq Angewandte Mathematik\grqq.}\label{tab2}
\begin{tabular}{| p{0.3\linewidth} | p{0.3\linewidth} | p{0.3\linewidth} |}\hline
Datum & Thema & Raum\\\hline
20.08.2008 & Graphentheorie & HS 3.13\\
01.10.2008 & Biomathematik & HS 1.05\\\hline
\end{tabular}
\end{table}

Hier wird auf die Formel \ref{Gl1} verwiesen.

\begin{align}
x = -\frac{p}{2}\pm\sqrt{\frac{p^2}{4}-q}\label{Gl1}
\end{align}
\begin{align}
x = -\frac{p}{2}\pm\sqrt{\frac{p^2}{4}-q}\label{Gl2}
\end{align}

\begin{lstlisting}[language=C++,name={1. Beispiel},label={sc:bsp:1}]
#include <iostream>

void SayHello(void)
{
    // Kommentar
    cout << "Hello World!" << endl;
}

int main(int argc, char **argv)
{
    SayHello();
    return 0;
}
\end{lstlisting}

Literaturverweise sollten automatisch verwaltet werden, vor allem, wenn es viele Quellenverweise gibt. Beispiele sind  \cite{Ko05a}, \cite{Ko05b}, \cite{MiGo05}, \cite{TeGo14}, \cite{HuHa07}, \cite{HuZi10}, \cite{ZiKu07}, \cite{He07}, \cite{SIE11}, \cite{SIE14}, \cite{ISO98}, \cite{ATM11}, \cite{Hu11}, \cite{Po10}. Das verwendete Zitierformat (bzw.~das Format des Literaturverzeichnisses) ist entspechend der Vorgaben der Studiengänge zu wählen.
Es wird dringend empfohlen, BibTeX~zu verwenden (wie in diesem Beispiel).

\chapter{Dritte Überschrift der Tiefe 1 (chapter)}
\begin{figure}[!htbp]
\centering
\includegraphics[width=0.5\linewidth]{PICs/buchruecken}
\caption{3. Beispiel für die Beschriftung eines Buchrückens.}\label{Abb3}
\end{figure}
\begin{figure}[!htbp]
\centering
\includegraphics[width=0.5\linewidth]{PICs/buchruecken}
\caption{4. Beispiel für die Beschriftung eines Buchrückens.}\label{Abb4}
\end{figure}


\begin{table}[!htbp]
\centering
\caption{3. Semesterplan der Lehrveranstaltung \glqq Angewandte Mathematik\grqq.}\label{tab3}
\begin{tabular}{| p{0.3\linewidth} | p{0.3\linewidth} | p{0.3\linewidth} |}\hline
Datum & Thema & Raum\\\hline
20.08.2008 & Graphentheorie & HS 3.13\\
01.10.2008 & Biomathematik & HS 1.05\\\hline
\end{tabular}
\end{table}
\begin{table}[!htbp]
\centering
\caption{4. Semesterplan der Lehrveranstaltung \glqq Angewandte Mathematik\grqq.}\label{tab4}
\begin{tabular}{| p{0.3\linewidth} | p{0.3\linewidth} | p{0.3\linewidth} |}\hline
Datum & Thema & Raum\\\hline
20.08.2008 & Graphentheorie & HS 3.13\\
01.10.2008 & Biomathematik & HS 1.05\\\hline
\end{tabular}
\end{table}

\begin{align}
x = -\frac{p}{2}\pm\sqrt{\frac{p^2}{4}-q}\label{Gl3}
\end{align}
\begin{align}
x = -\frac{p}{2}\pm\sqrt{\frac{p^2}{4}-q}\label{Gl4}
\end{align}
\begin{lstlisting}[language=C++,name={2. Beispiel},label={sc:bsp:2}]
#include <iostream>

void SayHello(void)
{
    // Kommentar
    cout << "Hello World!" << endl;
}

int main(int argc, char **argv)
{
    SayHello();
    return 0;
}
\end{lstlisting}

%
% Hier beginnen die Verzeichnisse.
%
\clearpage
\ifthenelse{\equal{\FHTWCitationType}{HARVARD}}{}{\bibliographystyle{gerabbrv}}
\bibliography{Literatur}
\clearpage

% Das Abbildungsverzeichnis
\listoffigures
\clearpage

% Das Tabellenverzeichnis
\listoftables
\clearpage

% Das Quellcodeverzeichnis
\listofcode
\clearpage

\phantomsection
\addcontentsline{toc}{chapter}{\listacroname}
\chapter*{\listacroname}
\begin{acronym}[XXXXX]
    \acro{ABC}[ABC]{Alphabet}
    \acro{WWW}[WWW]{world wide web}
    \acro{ROFL}[ROFL]{Rolling on floor laughing}
\end{acronym}

%
% Hier beginnt der Anhang.
%
\clearpage
\appendix
\chapter{Anhang A}
\clearpage
\chapter{Anhang B}
\end{document}